\documentclass[12pt,a4paper]{jsarticle}
\usepackage[margin=20truemm]{geometry}

% 節番号を「第1節」形式に
\renewcommand{\thesection}{第\arabic{section}節}

\usepackage{amsmath}
\usepackage{ascmac}
\usepackage{amsthm}
\usepackage{amsfonts}
\usepackage{latexsym}
\usepackage{bm}
\usepackage{mathtools}
\usepackage{empheq}
\usepackage{amssymb}
\usepackage[dvipdfmx]{graphicx}
\usepackage{tikz}
\usepackage{textcomp}
\usepackage{enumitem}
\usepackage[most]{tcolorbox}
\usepackage[dvipdfmx]{xcolor}
\usepackage{environ}
\usetikzlibrary{intersections,calc,arrows.meta}

% 行間を少し広げる
\linespread{1.15}

% 分数を常に大きく表示
\everymath{\displaystyle}

% 数式環境前後の空白を調整(ドキュメント開始時に適用)
\AtBeginDocument{%
  \setlength{\abovedisplayskip}{1pt}%
  \setlength{\belowdisplayskip}{6pt}%
  \setlength{\abovedisplayshortskip}{1pt}%
  \setlength{\belowdisplayshortskip}{3pt}%
}

% 平均記号のバーを長くするコマンド
\newcommand{\mean}[1]{\overline{#1}}

% 分数の穴埋めコマンド(下線なし)
% 分数用コマンド(表専用 - 分子を空欄にしてレイアウト保持)
\newcommand{\fracblank}[2]{%
  \ifshowanswer
    \textcolor{blue}{\frac{#1}{#2}}%  解答版:青色
  \else
    \frac{\phantom{#1}}{#2}%  空欄版:分子を透明化、レイアウト保持
  \fi
}

% enumerate環境のデフォルト設定((1)、(2)形式)
\setlist[enumerate,1]{label=(\arabic*), leftmargin=*, labelsep=0.5em}
\setlist[enumerate,2]{label=(\alph*), leftmargin=*, labelsep=0.5em}

\usepackage[normalem]{ulem} % 下線用
\usepackage{stackengine} % 追加

% 丸囲み文字コマンド
\newcommand{\ctext}[1]{\raise0.2ex\hbox{\textcircled{\scriptsize{#1}}}}

% 定理スタイルのカスタマイズ(最後のピリオドを削除)
\newtheoremstyle{myplain}
  {3pt}   % Space above
  {3pt}   % Space below
  {\itshape}  % Body font
  {}  % Indent amount
  {\bfseries} % Theorem head font
  {}  % Punctuation after theorem head (空にしてピリオドを削除)
  {.5em}  % Space after theorem head
  {}  % Theorem head spec

\newtheoremstyle{mydefinition}
  {3pt}   % Space above
  {3pt}   % Space below
  {}  % Body font
  {}  % Indent amount
  {\bfseries} % Theorem head font
  {}  % Punctuation after theorem head (空にしてピリオドを削除)
  {.5em}  % Space after theorem head
  {}  % Theorem head spec

\theoremstyle{mydefinition}
\newtheorem{definition}{Definition}
\newtheorem*{definition*}{Definition}
\theoremstyle{myplain}
\newtheorem{theorem}{Theorem}
\newtheorem*{theorem*}{Theorem}

% ========================================
% 問題環境の設定
% ========================================
% 問題番号カウンター(単純な連番: 1, 2, 3, ...)
\newcounter{problemcounter}
\renewcommand{\theproblemcounter}{\arabic{problemcounter}}

% 問題環境(枠なし、シンプル)
\newenvironment{problem}[1][\relax]{%
  \refstepcounter{problemcounter}%
  \vspace{2em}%
  \noindent%
  \textbf{問題\theproblemcounter}%
  \ifx#1\relax\else~\textbf{#1}\fi%
  \par\vspace{0.8em}%
  \noindent%
}{%
  \par\vspace{2em}%
}

% 問題環境(枠付き)
\newenvironment{problembox}[1][\relax]{%
  \refstepcounter{problemcounter}%
  \vspace{2em}%
  \begin{itembox}[l]{\textbf{問題\theproblemcounter}%
  \ifx#1\relax\else~\textbf{#1}\fi}%
}{%
  \end{itembox}%
  \vspace{1.5em}%
}

% 例題環境(枠付き)- 【】内の番号を使用
\newenvironment{example}[1][\relax]{%
  \vspace{2em}%
  \begin{itembox}[l]{\textbf{例題}%
  \ifx#1\relax\else~\textbf{#1}\fi}%
}{%
  \end{itembox}%
  \vspace{1.5em}%
}

% 練習問題環境(枠付き)- 【】内の番号を使用
\newenvironment{exercise}[1][\relax]{%
  \vspace{2em}%
  \begin{itembox}[l]{\textbf{練習問題}%
  \ifx#1\relax\else~\textbf{#1}\fi}%
}{%
  \end{itembox}%
  \vspace{1.5em}%
}

% 解答環境
\newenvironment{solution}{%
  \vspace{1em}%
  \noindent%
  \ifshowanswer%
    \textbf{【解答】}\\%
  \fi%
}{%
  \par\vspace{1em}%
}

% 問題専用の解答環境(常に空欄、手書きスペース確保)
% 使い方: \begin{problemsolution}[高さ] ... \end{problemsolution}
% デフォルト高さは15cm
% ※最後の空白は手動で \vspace{5\baselineskip} を追加してください
% 空欄時: 解答内容を白色で出力し、証明終了記号(□)を同じ位置に配置
\NewEnviron{problemsolution}[1][15cm]{%
  \begin{proof}\mbox{}\\%
  \ifshowanswer%
    {\color{blue}\BODY}%
  \else%
    {\color{white}\BODY}%
  \fi%
  \end{proof}%
}

% 別解環境(解答表示時のみ表示、ピリオドなし)
\NewEnviron{altproof}{%
  \ifshowanswer%
    \par\vspace{1em}%
    \noindent\textbf{【別解】}\par%
    {\color{blue}\BODY\par\hfill$\square$}%
    \vspace{1em}%
  \fi%
}

% ========================================
% 解答表示の切り替え設定
% ========================================
% 解答を表示したい場合:\showanswertrue
% 解答を隠したい場合:\showanswerfalse
\newif\ifshowanswer
% ←ここを \showanswertrue に変更すると解答が表示されます
\showanswertrue
% 穴埋めコマンドの修正版(upLaTeX対応)
\newcommand{\Blank}[2][3cm]{%
  ~%
  \ifshowanswer
    \stackon[1pt]{\uline{\makebox[#1][l]{\hphantom{あ}}}}{{\normalsize\sffamily #2}}%
  \else
    \uline{\makebox[#1][l]{\hphantom{あ}}}%
  \fi
  ~%
}

% 正解に合わせて自動で空欄を作る(em単位版)
% 使用法: \fitblank[答え]{幅} または \fitblank{幅}(答えなしの場合)
% 太文字版: \fitblankbf[答え]{幅}
\newcommand{\fitblank}[2][\relax]{%
  \ifx#1\relax
    % 答えが指定されていない場合(従来の使用法)
    \underline{\hspace*{#2em}}%
  \else
    % 答えが指定されている場合
    \ifshowanswer
      \underline{\makebox[#2em][c]{#1}}%
    \else
      \underline{\hspace*{#2em}}%
    \fi
  \fi
}

% 太文字版の空欄コマンド
\newcommand{\fitblankbf}[2][\relax]{%
  \ifx#1\relax
    % 答えが指定されていない場合(従来の使用法)
    \underline{\hspace*{#2em}}%
  \else
    % 答えが指定されている場合
    \ifshowanswer
      \underline{\makebox[#2em][c]{\textbf{#1}}}%
    \else
      \underline{\hspace*{#2em}}%
    \fi
  \fi
}

% 数式と文字の両方を太字にするコマンド
\newcommand{\fitblankbold}[2][\relax]{%
  \ifx#1\relax
    % 答えが指定されていない場合(従来の使用法)
    \underline{\hspace*{#2em}}%
  \else
    % 答えが指定されている場合
    \ifshowanswer
      \underline{\makebox[#2em][c]{\boldmath\textbf{#1}}}%
    \else
      \underline{\hspace*{#2em}}%
    \fi
  \fi
}

% ========================================
% tcolorbox環境(背景パターンで区別)
% ========================================

% 定義環境(赤系:薄い赤背景 + 赤枠)- 自動番号付き
% 注: #1はラベル(オプション)、#2はタイトル(必須)
\newtcolorbox{definitionbox}[2][]{
    enhanced,
    colback=red!5,
    colframe=red!70!black,
    boxrule=0.8pt,
    sharp corners,
    left=10pt, right=10pt, top=8pt, bottom=8pt,
    fonttitle=\bfseries,
    code={\refstepcounter{definition}\if\relax\detokenize{#1}\relax\else\label{#1}\fi}, % #1が空でない場合のみラベル設定
    title={Definition \thedefinition\if\relax\detokenize{#2}\relax\else\quad\textbf{#2}\fi}
}

% 定理環境(青系:薄い青背景 + 青枠)- 自動番号付き
% 注: #1はラベル(オプション)、#2はタイトル(必須)
\newtcolorbox{theorembox}[2][]{
    enhanced,
    colback=blue!5,
    colframe=blue!70!black,
    boxrule=1pt,
    sharp corners,
    left=10pt, right=10pt, top=8pt, bottom=8pt,
    fonttitle=\bfseries,
    code={\refstepcounter{theorem}\if\relax\detokenize{#1}\relax\else\label{#1}\fi}, % #1が空でない場合のみラベル設定
    title={Theorem \thetheorem\if\relax\detokenize{#2}\relax\else\quad\textbf{#2}\fi}
}

% 問題環境(濃い灰色背景 + 太線)
\newtcolorbox{problembox2}[1][]{
    enhanced,
    colback=black!10,
    colframe=black,
    boxrule=1.2pt,
    sharp corners,
    left=10pt, right=10pt, top=8pt, bottom=8pt,
    fonttitle=\bfseries,
    title=#1
}

% ========================================
% 手書き用空欄コマンド(下線なし、スペース確保)
% ========================================

% 解答ブロック全体の表示/非表示
% 使い方: \answerblock{解答内容}
\newcommand{\answerblock}[1]{%
  \ifshowanswer
    \textcolor{blue}{#1}%  解答を青色で表示
  \else
    % 空欄時は何も表示しない
  \fi
}

% ========================================
% answer用コマンド(レイアウト基準は答えの幅、空欄時も幅を確保)
% ========================================

% インライン用(テキスト・数式両用)
% 使い方: \answertext{答え}
% 空欄時: 答えの幅 + 両端に1em(1文字分)の余白
\newcommand{\answertext}[1]{%
  \ifshowanswer
    \textcolor{blue}{#1}%  解答を青色で表示
  \else
    \hspace{1em}\phantom{#1}\hspace{1em}%  空欄時は答えの幅 + 両端1em
  \fi
}

% 数式ブロック用(align環境など内部で使用)
% 使い方: \answermath{数式}
% 空欄時: 答えの幅 + 両端に2em(2文字分)の余白
\newcommand{\answermath}[1]{%
  \ifshowanswer
    {\color{blue}#1}%  数式を青色で表示
  \else
    \hspace{2em}\phantom{#1}\hspace{2em}%  空欄時は答えの幅 + 両端2em
  \fi
}

% 表専用コマンド(空欄にしてレイアウト保持)
% 使い方: \answertable{答え}
% 空欄版:透明化してレイアウト保持
\newcommand{\answertable}[1]{%
  \ifshowanswer
    \textcolor{blue}{#1}%  解答版:青色で表示
  \else
    \phantom{#1}%  空欄版:透明化、レイアウト保持
  \fi
}

% ========================================
% blank用コマンド(解答時も余白あり版)
% ※通常は answer用コマンドを使用してください
% ========================================

% インライン用(テキスト・数式両用)
% 使い方: \blanktext{答え}
% 解答時・空欄時とも: 答えの幅 + 両端に0.5em空白
\newcommand{\blanktext}[1]{%
  \ifshowanswer
    \hspace{0.5em}\textcolor{blue}{#1}\hspace{0.5em}%
  \else
    \hspace{0.5em}\phantom{#1}\hspace{0.5em}%
  \fi
}

% 数式ブロック用(align環境など内部で使用)
% 使い方: \blankmath{数式}
% 解答時・空欄時とも: 答えの幅 + 左1.5em、右2.5em空白
\newcommand{\blankmath}[1]{%
  \ifshowanswer
    \hspace{1.5em}{\color{blue}#1}\hspace{2.5em}%
  \else
    \hspace{1.5em}\phantom{#1}\hspace{2.5em}%
  \fi
}

% 手書きスペース確保版(指定した高さの空白を作る)
% 使い方: \answerspace[高さ]{解答内容}
% 高さのデフォルトは5cm
\newcommand{\answerspace}[2][5cm]{%
  \ifshowanswer
    \textcolor{blue}{#2}%  解答を青色で表示
  \else
    \vspace{#1}%  指定した高さの空白を確保
  \fi
}

% 既存の青色空欄コマンド(下線付き版も追加)
% 青色版の fitblank
\newcommand{\fitblankblue}[2][\relax]{%
  \ifx#1\relax
    % 答えが指定されていない場合
    \underline{\hspace*{#2em}}%
  \else
    % 答えが指定されている場合
    \ifshowanswer
      \textcolor{blue}{\underline{\makebox[#2em][c]{#1}}}%
    \else
      \underline{\hspace*{#2em}}%
    \fi
  \fi
}

% 青色太文字版
\newcommand{\fitblankbfblue}[2][\relax]{%
  \ifx#1\relax
    \underline{\hspace*{#2em}}%
  \else
    \ifshowanswer
      \textcolor{blue}{\underline{\makebox[#2em][c]{\textbf{#1}}}}%
    \else
      \underline{\hspace*{#2em}}%
    \fi
  \fi
}

% 分数の青色版
% 分数用コマンド(表専用 - 分子を空欄にしてレイアウト保持)
\newcommand{\fracblankblue}[2]{%
  \ifshowanswer
    \textcolor{blue}{\frac{#1}{#2}}%  解答版:青色
  \else
    \frac{\phantom{#1}}{#2}%  空欄版:分子を透明化、レイアウト保持
  \fi
}


% subsection番号を「1.1」形式に(thesectionの影響を受けないように)
\renewcommand{\thesubsection}{\arabic{section}.\arabic{subsection}}

% 定理・定義番号をsectionごとにリセット
\numberwithin{theorem}{section}
\numberwithin{definition}{section}

% 定理番号を「1.1」形式に
\renewcommand{\thetheorem}{\arabic{section}.\arabic{theorem}}
\renewcommand{\thedefinition}{\arabic{section}.\arabic{definition}}

% カウンターの初期値を設定
\setcounter{section}{0}
\setcounter{subsection}{0}

% 定義・定理番号を1から始める
\setcounter{definition}{0}
\setcounter{theorem}{0}

% 式番号を「1.1」形式に(sectionごとにリセット)
\numberwithin{equation}{section}
\renewcommand{\theequation}{\arabic{section}.\arabic{equation}}

\title{【数学B】 統計的な推測}                % タイトル
\author{}                     % 著者(空白でOK)
\date{}                       % 日付(空白)

\begin{document}
\begin{center}
{\LARGE 【中学数学】 第3章 図形の性質と合同}

体系数学1/数研出版/幾何編
\end{center}

% --- 著者と日付を右寄せで表示 ---
\vspace{5mm}
\hfill Riku Sugawara \\
\hfill 11.2025

% ========================================
% 各セクションの読み込み
% ========================================

% 第1節 平行線と角
\section{平行線と角}

この単元で学習する角の種類は3種類である。それは\fitblankbold[対頂角]{5}、\fitblankbold[同位角]{5}、\fitblankbold[錯角]{5}である。

    \begin{definitionbox}[def:対頂角]{\textbf{対頂角}}
        2直線が交わるとき、その交点の周りには4つの角ができる。このうち
        \fitblankbold[向かい合っている]{10} 2つの角を \fitblankbold[対頂角]{5} という。
    \end{definitionbox}
    \vspace{1em}

対頂角を図で表すと以下のようになる。また、そこから次のことが分かる。

\newpage

    \begin{theorembox}[thm:対頂角]{\textbf{対頂角の性質}}
        対頂角は \fitblankbold[いつでも]{5} 等しい。
    \end{theorembox}
    \vspace{1em}

    \begin{definitionbox}[def:同位角]{\textbf{同位角}}
        2直線に1つの直線が交わるとき、その交点の同じ側にできる角を \fitblankbold[同位角]{5} という。
    \end{definitionbox}
    \vspace{1em}

    \begin{definitionbox}[def:錯角]{\textbf{錯角}}
        2直線に1つの直線が交わるとき、その交点の反対側にできる角を \fitblankbold[錯角]{5} という。
    \end{definitionbox}
    \vspace{1em}

同位角と錯角を図で表すと以下のようになる。また、そこから次のことが分かる。

錯角は \fitblankbold[Z]{3}、その鏡文字の \fitblankbold[S]{3} の内側にできる角と考えると良い。

\newpage

    \begin{theorembox}[thm:同位角と錯角の性質①]{\textbf{同位角と錯角の性質①}}
        同位角と錯角は、いつでも等しい \fitblankbold[とは限らない]{8} 。
    \end{theorembox}
    \vspace{1em}

    \begin{theorembox}[thm:同位角と錯角の性質②]{\textbf{同位角と錯角の性質②}}
        2直線が \fitblankbold[平行]{5} ならば同位角、錯角はそれぞれ \fitblankbold[等しい]{5}。

        また、これはその逆も成り立つ。すなわち、

        同位角、錯角が等しい ならば 2直線は \fitblankbold[平行]{5} である。
    \end{theorembox}
    \vspace{1em}

Theorem \ref{thm:同位角と錯角の性質②} を図示すると以下のようになる。

\newpage


% 第2節 多角形の内角と外角
\section{多角形の内角と外角}

まずは三角形について考える。
%三角形
    \begin{definitionbox}[def:三角形の内角と外角]{\textbf{三角形の内角と外角}}
        三角形の3つの辺がつくる三角形の内部にある角のことを \fitblankbf[内角]{5} という。
        これに対し、1つの辺とそれと隣り合う辺の延長がつくる角のことを
        \fitblankbf[外角]{5} という。
    \end{definitionbox}
    \vspace{1.5em}

また、三角形の内角と外角に対して次のことがいえる。

\vspace{1.5em}
    \begin{theorembox}[thm:三角形の内角と外角の性質]{\textbf{三角形の内角と外角の性質}}
        三角形の3つの内角の和は \fitblankbf[180°]{5} である。

        三角形の1つの外角は、\fitblankbf[それと隣り合わない2つの]{15} 内角の和に等しい。
    \end{theorembox}
    \vspace{1em}

三角形の内角と外角、それらの性質を図で表すと上のようになる。
\vspace{15em}

次に、多角形について考える。
%多角形
    \begin{theorembox}[thm:多角形の内角の和]{\textbf{多角形の内角の和}}
        多角形の内角の和は以下の式で表すことができる。

        \begin{align*}
            \bm{n} \textbf{角形の内角の和} &\bm{=} \fitblankbf[180° \times (n - 2)]{15}
        \end{align*}
    \end{theorembox}
    \vspace{1em}

    \begin{theorembox}[thm:多角形の外角の和]{\textbf{多角形の外角の和}}
        $n$ 角形の外角の和は \fitblankbf[360°]{5} である。
    \end{theorembox}
    \vspace{1em}

これらの式は、左辺と右辺をセットで覚え、使ってもらいたい。

\newpage


% 第3節 三角形の合同
\section{三角形の合同}

%合同
    \begin{definitionbox}[def:合同]{\textbf{合同}}
        一方の図形を平行移動、回転移動などすることで他方の図形にぴったりと重ね合わせることができるとき
        それらの図形を \fitblankbold[合同]{5} という。
    \end{definitionbox}
    \vspace{1em}

    一言で言うと、\fitblankbold[全く同じ]{5} の図形のことである。
    したがって、

    \begin{enumerate}
        \item 合同な図形では 対応する \fitblankbold[線分の長さ]{8} はそれぞれ等しい。
        \item 合同な図形では 対応する \fitblankbold[角の大きさ]{8} はそれぞれ等しい。
    \end{enumerate}
    \vspace{1em}

    \begin{definitionbox}[def:合同な図形の表し方]{\textbf{合同な図形の表し方}}
        2つの図形が合同であることを記号 $\fitblankbold[\equiv]{3}$ を用いて表す。

        例えば、$\triangle ABC$ と $\triangle DEF$ が合同であることを $\fitblankbold[\triangle ABC \equiv \triangle DEF]{8}$ と表し、
        「三角形 ABC 合同 三角形 DEF」と読む。
        また、記号 $\equiv$ を用いるときは、頂点の順番を \fitblankbold[対応する順]{5} で書く。
    \end{definitionbox}
    \vspace{1em}

    特に三角形においては、合同になるための条件が知られている。これを \fitblankbold[合同条件]{5} といい、次の3つがある。
    合同条件は証明中に使い、\uwave{一言一句違わずに} 使わなければならない。

%合同条件
    \begin{theorembox}[thm:合同条件]{\textbf{合同条件}}
        2つの三角形が合同であるための条件は以下の3つである。

        \begin{enumerate}
            \item \fitblankbold[3組の辺がそれぞれ等しい]{10}
            \item \fitblankbold[2組の辺とその間の角がそれぞれ等しい]{10}
            \item \fitblankbold[1組の辺とその両端の角がそれぞれ等しい]{10}
        \end{enumerate}
    \end{theorembox}
    \vspace{1em}















\newpage

% 第4節 証明
\section{証明}

証明の流れち型を身につけてもらいたい。学校の型と異なるかとは思うが、ここでは私がオススメする型を紹介する。
特にこだわりがない場合は、この型を是非使ってもらいたい。

ここで使用する流れや型は2年次で学習する「合同な図形」の証明においても、3年次で学習する「相似な図形」の証明においても使用できる汎用的なものである。

%証明の流れ
    \begin{theorembox}[thm:証明の流れ]{\textbf{証明の流れ}}
        \label{thm:証明の流れ}
        証明の流れは以下のようになる。

        \begin{enumerate}
            \item \fitblankbf[証明したい図形]{10} を確認し、一方を\fitblankbf[赤い三角形]{8}で、もう一方を\fitblankbf[青い三角形]{8}で囲む。
            \item 赤い三角形と青い三角形を \fitblankbf[抜き出し]{8} $^{*}$、\fitblankbf[仮定]{8} や \fitblankbf[条件]{8} を描き込む。
            \item(2)で抜き出した図形をもとに \fitblankbf[合同条件]{8} を確定させる。$^{**}$
        \end{enumerate}
    \end{theorembox}
    \vspace{1em}

    Theorem \ref{thm:証明の流れ} で証明の下準備は完了!証明は書き始める前に終えてなければならない。
    したがって、Theorem \ref{thm:証明の流れ} を使って証明の型にはめていくことになる。
    \vspace{0.5em}

    \begin{theorembox}[thm:証明の型]{\textbf{証明の型}}
        \label{thm:証明の型}
        証明の型は以下のようになる。ここでは \textcolor{red}{\underline{$\triangle ABC$}} と \textcolor{blue}{\underline{$\triangle DEF$}} が合同であることを証明する。

        \begin{proof}
            \setlength{\baselineskip}{1.5\baselineskip}
            \mbox{}\\
            \textcolor{red}{\fitblankbf[\triangle ABC]{8}} と \textcolor{blue}{\fitblankbf[\triangle DEF]{8}} において

            \vspace{0.3em}
            \textcolor{black}{\uwave{\hspace{2em}根拠①\hspace{3em}}} より、

            \vspace{0.3em}
            \makebox[\linewidth][s]{\textcolor{red}{\fitblankbf[AB]{8}} = \textcolor{blue}{\fitblankbf[DE]{8}} \hfill ・・・①}

            \vspace{0.3em}
            \textcolor{black}{\uwave{\hspace{2em}根拠②\hspace{3em}}} ので、

            \vspace{0.3em}
            \makebox[\linewidth][s]{\textcolor{red}{\fitblankbf[\angle BAC]{8}} = \textcolor{blue}{\fitblankbf[\angle EDF]{8}} \hfill ・・・②}

            \vspace{0.3em}
            \textcolor{black}{\uwave{\hspace{2em}根拠③\hspace{3em}}} ので、

            \vspace{0.3em}
            \makebox[\linewidth][s]{\textcolor{red}{\fitblankbf[CA]{8}} = \textcolor{blue}{\fitblankbf[FD]{8}} \hfill ・・・③}

            \vspace{0.3em}
            ①、②、③より、\textcolor{black}{\fitblankbf[合同条件]{15}} ので、

            \vspace{0.3em}
            \textcolor{red}{\fitblankbf[\triangle ABC]{8}} $\equiv$ \textcolor{blue}{\fitblankbf[\triangle DEF]{8}}
        \end{proof}
    \end{theorembox}
\vfill

\noindent
\rule{\textwidth}{0.4pt}

\noindent
$^{*}$ 図形を抜き出す際には \textbf{ 形は気にせず } 抜き出してもらって構わない。しかし、2つの三角形とも \textbf{ 同じ形 } で
\textbf{ 対応する頂点 } で抜き出すこと。

$^{**}$ 合同条件の確定には \textbf{ 結論 } を使ってはならない。
\newpage





\end{document}

