% 第2節 多角形の内角と外角
\section{多角形の内角と外角}

まずは三角形について考える。
%三角形
    \begin{definitionbox}[def:三角形の内角と外角]{\textbf{三角形の内角と外角}}
        三角形の3つの辺がつくる三角形の内部にある角のことを \fitblankbf[内角]{5} という。
        これに対し、1つの辺とそれと隣り合う辺の延長がつくる角のことを
        \fitblankbf[外角]{5} という。
    \end{definitionbox}
    \vspace{1.5em}

また、三角形の内角と外角に対して次のことがいえる。

\vspace{1.5em}
    \begin{theorembox}[thm:三角形の内角と外角の性質]{\textbf{三角形の内角と外角の性質}}
        三角形の3つの内角の和は \fitblankbf[180°]{5} である。

        三角形の1つの外角は、\fitblankbf[それと隣り合わない2つの]{15} 内角の和に等しい。
    \end{theorembox}
    \vspace{1em}

三角形の内角と外角、それらの性質を図で表すと上のようになる。
\vspace{15em}

次に、多角形について考える。
%多角形
    \begin{theorembox}[thm:多角形の内角の和]{\textbf{多角形の内角の和}}
        多角形の内角の和は以下の式で表すことができる。

        \begin{align*}
            \bm{n} \textbf{角形の内角の和} &\bm{=} \fitblankbf[180° \times (n - 2)]{15}
        \end{align*}
    \end{theorembox}
    \vspace{1em}

    \begin{theorembox}[thm:多角形の外角の和]{\textbf{多角形の外角の和}}
        $n$ 角形の外角の和は \fitblankbf[360°]{5} である。
    \end{theorembox}
    \vspace{1em}

これらの式は、左辺と右辺をセットで覚え、使ってもらいたい。

\newpage

