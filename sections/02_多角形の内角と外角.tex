% 第2節 多角形の内角と外角
\section{多角形の内角と外角}

まずは三角形について考える。

    \begin{definitionbox}[def:三角形の内角と外角]{\textbf{三角形の内角と外角}}
        三角形の3つの辺がつくる三角形の内部にある角のことを \fitblankbold[内角]{5} という。
        これに対し、1つの辺とそれと隣り合う辺の延長がつくる角のことを
        \fitblankbold[外角]{5} という。
    \end{definitionbox}
    \vspace{1.5em}

また、三角形の内角と外角に対して次のことがいえる。

\vspace{1.5em}
    \begin{theorembox}[thm:三角形の内角と外角の性質]{\textbf{三角形の内角と外角の性質}}
        三角形の3つの内角の和は \fitblankbold[180°]{5} である。

        三角形の1つの外角は、\fitblankbold[それと隣り合わない2つの]{15} 内角の和に等しい。
    \end{theorembox}
    \vspace{1em}

三角形の内角と外角、それらの性質を図で表すと上のようになる。
\newpage

次に、多角形について考える。


\newpage

