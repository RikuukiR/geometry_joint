% 第3節 三角形の合同
\section{三角形の合同}

%合同
    \begin{definitionbox}[def:合同]{\textbf{合同}}
        一方の図形を平行移動、回転移動などすることで他方の図形にぴったりと重ね合わせることができるとき
        それらの図形を \fitblankbold[合同]{5} という。
    \end{definitionbox}
    \vspace{1em}

    一言で言うと、\fitblankbold[全く同じ]{5} の図形のことである。
    したがって、

    \begin{enumerate}
        \item 合同な図形では 対応する \fitblankbold[線分の長さ]{8} はそれぞれ等しい。
        \item 合同な図形では 対応する \fitblankbold[角の大きさ]{8} はそれぞれ等しい。
    \end{enumerate}
    \vspace{1em}

    \begin{definitionbox}[def:合同な図形の表し方]{\textbf{合同な図形の表し方}}
        2つの図形が合同であることを記号 $\fitblankbold[\equiv]{3}$ を用いて表す。

        例えば、$\triangle ABC$ と $\triangle DEF$ が合同であることを $\fitblankbold[\triangle ABC \equiv \triangle DEF]{8}$ と表し、
        「三角形 ABC 合同 三角形 DEF」と読む。
        また、記号 $\equiv$ を用いるときは、頂点の順番を \fitblankbold[対応する順]{5} で書く。
    \end{definitionbox}
    \vspace{1em}

    特に三角形においては、合同になるための条件が知られている。これを \fitblankbold[合同条件]{5} といい、次の3つがある。
    合同条件は証明中に使い、\uwave{一言一句違わずに} 使わなければならない。

%合同条件
    \begin{theorembox}[thm:合同条件]{\textbf{合同条件}}
        2つの三角形が合同であるための条件は以下の3つである。

        \begin{enumerate}
            \item \fitblankbold[3組の辺がそれぞれ等しい]{10}
            \item \fitblankbold[2組の辺とその間の角がそれぞれ等しい]{10}
            \item \fitblankbold[1組の辺とその両端の角がそれぞれ等しい]{10}
        \end{enumerate}
    \end{theorembox}
    \vspace{1em}















\newpage
