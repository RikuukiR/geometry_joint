% 第1節 平行線と角
\section{平行線と角}

この単元で学習する角の種類は3種類である。それは\fitblankbold[対頂角]{5}、\fitblankbold[同位角]{5}、\fitblankbold[錯角]{5}である。

    \begin{definitionbox}[def:対頂角]{\textbf{対頂角}}
        2直線が交わるとき、その交点の周りには4つの角ができる。このうち
        \fitblankbold[向かい合っている]{10} 2つの角を \fitblankbold[対頂角]{5} という。
    \end{definitionbox}
    \vspace{1em}

対頂角を図で表すと以下のようになる。また、そこから次のことが分かる。

\newpage

    \begin{theorembox}[thm:対頂角]{\textbf{対頂角の性質}}
        対頂角は \fitblankbold[いつでも]{5} 等しい。
    \end{theorembox}
    \vspace{1em}

    \begin{definitionbox}[def:同位角]{\textbf{同位角}}
        2直線に1つの直線が交わるとき、その交点の同じ側にできる角を \fitblankbold[同位角]{5} という。
    \end{definitionbox}
    \vspace{1em}

    \begin{definitionbox}[def:錯角]{\textbf{錯角}}
        2直線に1つの直線が交わるとき、その交点の反対側にできる角を \fitblankbold[錯角]{5} という。
    \end{definitionbox}
    \vspace{1em}

同位角と錯角を図で表すと以下のようになる。また、そこから次のことが分かる。

錯角は \fitblankbold[Z]{3}、その鏡文字の \fitblankbold[S]{3} の内側にできる角と考えると良い。

\newpage

    \begin{theorembox}[thm:同位角と錯角の性質①]{\textbf{同位角と錯角の性質①}}
        同位角と錯角は、いつでも等しい \fitblankbold[とは限らない]{8} 。
    \end{theorembox}
    \vspace{1em}

    \begin{theorembox}[thm:同位角と錯角の性質②]{\textbf{同位角と錯角の性質②}}
        2直線が \fitblankbold[平行]{5} ならば同位角、錯角はそれぞれ \fitblankbold[等しい]{5}。

        また、これはその逆も成り立つ。すなわち、

        同位角、錯角が等しい ならば 2直線は \fitblankbold[平行]{5} である。
    \end{theorembox}
    \vspace{1em}

Theorem \ref{thm:同位角と錯角の性質②} を図示すると以下のようになる。

\newpage

