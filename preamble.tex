\usepackage{amsmath}
\usepackage{ascmac}
\usepackage{amsthm}
\usepackage{amsfonts}
\usepackage{latexsym}
\usepackage{bm}
\usepackage{mathtools}
\usepackage{empheq}
\usepackage{amssymb}
\usepackage[dvipdfmx]{graphicx}
\usepackage{tikz}
\usepackage{textcomp}
\usepackage{enumitem}
\usepackage[most]{tcolorbox}
\usepackage[dvipdfmx]{xcolor}
\usepackage{environ}
\usetikzlibrary{intersections,calc,arrows.meta}

% 行間を少し広げる
\linespread{1.15}

% 分数を常に大きく表示
\everymath{\displaystyle}

% 数式環境前後の空白を調整(ドキュメント開始時に適用)
\AtBeginDocument{%
  \setlength{\abovedisplayskip}{1pt}%
  \setlength{\belowdisplayskip}{6pt}%
  \setlength{\abovedisplayshortskip}{1pt}%
  \setlength{\belowdisplayshortskip}{3pt}%
}

% 平均記号のバーを長くするコマンド
\newcommand{\mean}[1]{\overline{#1}}

% 分数の穴埋めコマンド(下線なし)
% 分数用コマンド(表専用 - 分子を空欄にしてレイアウト保持)
\newcommand{\fracblank}[2]{%
  \ifshowanswer
    \textcolor{blue}{\frac{#1}{#2}}%  解答版:青色
  \else
    \frac{\phantom{#1}}{#2}%  空欄版:分子を透明化、レイアウト保持
  \fi
}

% enumerate環境のデフォルト設定((1)、(2)形式)
\setlist[enumerate,1]{label=(\arabic*), leftmargin=*, labelsep=0.5em}
\setlist[enumerate,2]{label=(\alph*), leftmargin=*, labelsep=0.5em}

\usepackage[normalem]{ulem} % 下線用
\usepackage{stackengine} % 追加

% 丸囲み文字コマンド
\newcommand{\ctext}[1]{\raise0.2ex\hbox{\textcircled{\scriptsize{#1}}}}

% 定理スタイルのカスタマイズ(最後のピリオドを削除)
\newtheoremstyle{myplain}
  {3pt}   % Space above
  {3pt}   % Space below
  {\itshape}  % Body font
  {}  % Indent amount
  {\bfseries} % Theorem head font
  {}  % Punctuation after theorem head (空にしてピリオドを削除)
  {.5em}  % Space after theorem head
  {}  % Theorem head spec

\newtheoremstyle{mydefinition}
  {3pt}   % Space above
  {3pt}   % Space below
  {}  % Body font
  {}  % Indent amount
  {\bfseries} % Theorem head font
  {}  % Punctuation after theorem head (空にしてピリオドを削除)
  {.5em}  % Space after theorem head
  {}  % Theorem head spec

\theoremstyle{mydefinition}
\newtheorem{definition}{Definition}
\newtheorem*{definition*}{Definition}
\theoremstyle{myplain}
\newtheorem{theorem}{Theorem}
\newtheorem*{theorem*}{Theorem}

% ========================================
% 問題環境の設定
% ========================================
% 問題番号カウンター(単純な連番: 1, 2, 3, ...)
\newcounter{problemcounter}
\renewcommand{\theproblemcounter}{\arabic{problemcounter}}

% 問題環境(枠なし、シンプル)
\newenvironment{problem}[1][\relax]{%
  \refstepcounter{problemcounter}%
  \vspace{2em}%
  \noindent%
  \textbf{問題\theproblemcounter}%
  \ifx#1\relax\else~\textbf{#1}\fi%
  \par\vspace{0.8em}%
  \noindent%
}{%
  \par\vspace{2em}%
}

% 問題環境(枠付き)
\newenvironment{problembox}[1][\relax]{%
  \refstepcounter{problemcounter}%
  \vspace{2em}%
  \begin{itembox}[l]{\textbf{問題\theproblemcounter}%
  \ifx#1\relax\else~\textbf{#1}\fi}%
}{%
  \end{itembox}%
  \vspace{1.5em}%
}

% 例題環境(枠付き)- 【】内の番号を使用
\newenvironment{example}[1][\relax]{%
  \vspace{2em}%
  \begin{itembox}[l]{\textbf{例題}%
  \ifx#1\relax\else~\textbf{#1}\fi}%
}{%
  \end{itembox}%
  \vspace{1.5em}%
}

% 練習問題環境(枠付き)- 【】内の番号を使用
\newenvironment{exercise}[1][\relax]{%
  \vspace{2em}%
  \begin{itembox}[l]{\textbf{練習問題}%
  \ifx#1\relax\else~\textbf{#1}\fi}%
}{%
  \end{itembox}%
  \vspace{1.5em}%
}

% 解答環境
\newenvironment{solution}{%
  \vspace{1em}%
  \noindent%
  \ifshowanswer%
    \textbf{【解答】}\\%
  \fi%
}{%
  \par\vspace{1em}%
}

% 問題専用の解答環境(常に空欄、手書きスペース確保)
% 使い方: \begin{problemsolution}[高さ] ... \end{problemsolution}
% デフォルト高さは15cm
% ※最後の空白は手動で \vspace{5\baselineskip} を追加してください
% 空欄時: 解答内容を白色で出力し、証明終了記号(□)を同じ位置に配置
\NewEnviron{problemsolution}[1][15cm]{%
  \begin{proof}\mbox{}\\%
  \ifshowanswer%
    {\color{blue}\BODY}%
  \else%
    {\color{white}\BODY}%
  \fi%
  \end{proof}%
}

% 別解環境(解答表示時のみ表示、ピリオドなし)
\NewEnviron{altproof}{%
  \ifshowanswer%
    \par\vspace{1em}%
    \noindent\textbf{【別解】}\par%
    {\color{blue}\BODY\par\hfill$\square$}%
    \vspace{1em}%
  \fi%
}

% ========================================
% 解答表示の切り替え設定
% ========================================
% 解答を表示したい場合:\showanswertrue
% 解答を隠したい場合:\showanswerfalse
\newif\ifshowanswer
% ←ここを \showanswertrue に変更すると解答が表示されます
\showanswertrue
% 穴埋めコマンドの修正版(upLaTeX対応)
\newcommand{\Blank}[2][3cm]{%
  ~%
  \ifshowanswer
    \stackon[1pt]{\uline{\makebox[#1][l]{\hphantom{あ}}}}{{\normalsize\sffamily #2}}%
  \else
    \uline{\makebox[#1][l]{\hphantom{あ}}}%
  \fi
  ~%
}

% 正解に合わせて自動で空欄を作る(em単位版)
% 使用法: \fitblank[答え]{幅} または \fitblank{幅}(答えなしの場合)
% 太文字版: \fitblankbf[答え]{幅}
\newcommand{\fitblank}[2][\relax]{%
  \ifx#1\relax
    % 答えが指定されていない場合(従来の使用法)
    \underline{\hspace*{#2em}}%
  \else
    % 答えが指定されている場合
    \ifshowanswer
      \underline{\makebox[#2em][c]{#1}}%
    \else
      \underline{\hspace*{#2em}}%
    \fi
  \fi
}

% 太文字版の空欄コマンド
\newcommand{\fitblankbf}[2][\relax]{%
  \ifx#1\relax
    % 答えが指定されていない場合(従来の使用法)
    \underline{\hspace*{#2em}}%
  \else
    % 答えが指定されている場合
    \ifshowanswer
      \underline{\makebox[#2em][c]{\textbf{#1}}}%
    \else
      \underline{\hspace*{#2em}}%
    \fi
  \fi
}

% 数式と文字の両方を太字にするコマンド
\newcommand{\fitblankbold}[2][\relax]{%
  \ifx#1\relax
    % 答えが指定されていない場合(従来の使用法)
    \underline{\hspace*{#2em}}%
  \else
    % 答えが指定されている場合
    \ifshowanswer
      \underline{\makebox[#2em][c]{\boldmath\textbf{#1}}}%
    \else
      \underline{\hspace*{#2em}}%
    \fi
  \fi
}

% ========================================
% tcolorbox環境(背景パターンで区別)
% ========================================

% 定義環境(赤系:薄い赤背景 + 赤枠)- 自動番号付き
% 注: #1はラベル(オプション)、#2はタイトル(必須)
\newtcolorbox{definitionbox}[2][]{
    enhanced,
    colback=red!5,
    colframe=red!70!black,
    boxrule=0.8pt,
    sharp corners,
    left=10pt, right=10pt, top=8pt, bottom=8pt,
    fonttitle=\bfseries,
    code={\refstepcounter{definition}\if\relax\detokenize{#1}\relax\else\label{#1}\fi}, % #1が空でない場合のみラベル設定
    title={Definition \thedefinition\if\relax\detokenize{#2}\relax\else\quad\textbf{#2}\fi}
}

% 定理環境(青系:薄い青背景 + 青枠)- 自動番号付き
% 注: #1はラベル(オプション)、#2はタイトル(必須)
\newtcolorbox{theorembox}[2][]{
    enhanced,
    colback=blue!5,
    colframe=blue!70!black,
    boxrule=1pt,
    sharp corners,
    left=10pt, right=10pt, top=8pt, bottom=8pt,
    fonttitle=\bfseries,
    code={\refstepcounter{theorem}\if\relax\detokenize{#1}\relax\else\label{#1}\fi}, % #1が空でない場合のみラベル設定
    title={Theorem \thetheorem\if\relax\detokenize{#2}\relax\else\quad\textbf{#2}\fi}
}

% 問題環境(濃い灰色背景 + 太線)
\newtcolorbox{problembox2}[1][]{
    enhanced,
    colback=black!10,
    colframe=black,
    boxrule=1.2pt,
    sharp corners,
    left=10pt, right=10pt, top=8pt, bottom=8pt,
    fonttitle=\bfseries,
    title=#1
}

% ========================================
% 手書き用空欄コマンド(下線なし、スペース確保)
% ========================================

% 解答ブロック全体の表示/非表示
% 使い方: \answerblock{解答内容}
\newcommand{\answerblock}[1]{%
  \ifshowanswer
    \textcolor{blue}{#1}%  解答を青色で表示
  \else
    % 空欄時は何も表示しない
  \fi
}

% ========================================
% answer用コマンド(レイアウト基準は答えの幅、空欄時も幅を確保)
% ========================================

% インライン用(テキスト・数式両用)
% 使い方: \answertext{答え}
% 空欄時: 答えの幅 + 両端に1em(1文字分)の余白
\newcommand{\answertext}[1]{%
  \ifshowanswer
    \textcolor{blue}{#1}%  解答を青色で表示
  \else
    \hspace{1em}\phantom{#1}\hspace{1em}%  空欄時は答えの幅 + 両端1em
  \fi
}

% 数式ブロック用(align環境など内部で使用)
% 使い方: \answermath{数式}
% 空欄時: 答えの幅 + 両端に2em(2文字分)の余白
\newcommand{\answermath}[1]{%
  \ifshowanswer
    {\color{blue}#1}%  数式を青色で表示
  \else
    \hspace{2em}\phantom{#1}\hspace{2em}%  空欄時は答えの幅 + 両端2em
  \fi
}

% 表専用コマンド(空欄にしてレイアウト保持)
% 使い方: \answertable{答え}
% 空欄版:透明化してレイアウト保持
\newcommand{\answertable}[1]{%
  \ifshowanswer
    \textcolor{blue}{#1}%  解答版:青色で表示
  \else
    \phantom{#1}%  空欄版:透明化、レイアウト保持
  \fi
}

% ========================================
% blank用コマンド(解答時も余白あり版)
% ※通常は answer用コマンドを使用してください
% ========================================

% インライン用(テキスト・数式両用)
% 使い方: \blanktext{答え}
% 解答時・空欄時とも: 答えの幅 + 両端に0.5em空白
\newcommand{\blanktext}[1]{%
  \ifshowanswer
    \hspace{0.5em}\textcolor{blue}{#1}\hspace{0.5em}%
  \else
    \hspace{0.5em}\phantom{#1}\hspace{0.5em}%
  \fi
}

% 数式ブロック用(align環境など内部で使用)
% 使い方: \blankmath{数式}
% 解答時・空欄時とも: 答えの幅 + 左1.5em、右2.5em空白
\newcommand{\blankmath}[1]{%
  \ifshowanswer
    \hspace{1.5em}{\color{blue}#1}\hspace{2.5em}%
  \else
    \hspace{1.5em}\phantom{#1}\hspace{2.5em}%
  \fi
}

% 手書きスペース確保版(指定した高さの空白を作る)
% 使い方: \answerspace[高さ]{解答内容}
% 高さのデフォルトは5cm
\newcommand{\answerspace}[2][5cm]{%
  \ifshowanswer
    \textcolor{blue}{#2}%  解答を青色で表示
  \else
    \vspace{#1}%  指定した高さの空白を確保
  \fi
}

% 既存の青色空欄コマンド(下線付き版も追加)
% 青色版の fitblank
\newcommand{\fitblankblue}[2][\relax]{%
  \ifx#1\relax
    % 答えが指定されていない場合
    \underline{\hspace*{#2em}}%
  \else
    % 答えが指定されている場合
    \ifshowanswer
      \textcolor{blue}{\underline{\makebox[#2em][c]{#1}}}%
    \else
      \underline{\hspace*{#2em}}%
    \fi
  \fi
}

% 青色太文字版
\newcommand{\fitblankbfblue}[2][\relax]{%
  \ifx#1\relax
    \underline{\hspace*{#2em}}%
  \else
    \ifshowanswer
      \textcolor{blue}{\underline{\makebox[#2em][c]{\textbf{#1}}}}%
    \else
      \underline{\hspace*{#2em}}%
    \fi
  \fi
}

% 分数の青色版
% 分数用コマンド(表専用 - 分子を空欄にしてレイアウト保持)
\newcommand{\fracblankblue}[2]{%
  \ifshowanswer
    \textcolor{blue}{\frac{#1}{#2}}%  解答版:青色
  \else
    \frac{\phantom{#1}}{#2}%  空欄版:分子を透明化、レイアウト保持
  \fi
}
